\documentclass[11pt]{article}

    \usepackage[breakable]{tcolorbox}
    \usepackage{parskip} % Stop auto-indenting (to mimic markdown behaviour)
    
    \usepackage{iftex}
    \ifPDFTeX
    	\usepackage[T1]{fontenc}
    	\usepackage{mathpazo}
    \else
    	\usepackage{fontspec}
    \fi

    % Basic figure setup, for now with no caption control since it's done
    % automatically by Pandoc (which extracts ![](path) syntax from Markdown).
    \usepackage{graphicx}
    % Maintain compatibility with old templates. Remove in nbconvert 6.0
    \let\Oldincludegraphics\includegraphics
    % Ensure that by default, figures have no caption (until we provide a
    % proper Figure object with a Caption API and a way to capture that
    % in the conversion process - todo).
    \usepackage{caption}
    \DeclareCaptionFormat{nocaption}{}
    \captionsetup{format=nocaption,aboveskip=0pt,belowskip=0pt}

    \usepackage{float}
    \floatplacement{figure}{H} % forces figures to be placed at the correct location
    \usepackage{xcolor} % Allow colors to be defined
    \usepackage{enumerate} % Needed for markdown enumerations to work
    \usepackage{geometry} % Used to adjust the document margins
    \usepackage{amsmath} % Equations
    \usepackage{amssymb} % Equations
    \usepackage{textcomp} % defines textquotesingle
    % Hack from http://tex.stackexchange.com/a/47451/13684:
    \AtBeginDocument{%
        \def\PYZsq{\textquotesingle}% Upright quotes in Pygmentized code
    }
    \usepackage{upquote} % Upright quotes for verbatim code
    \usepackage{eurosym} % defines \euro
    \usepackage[mathletters]{ucs} % Extended unicode (utf-8) support
    \usepackage{fancyvrb} % verbatim replacement that allows latex
    \usepackage{grffile} % extends the file name processing of package graphics 
                         % to support a larger range
    \makeatletter % fix for old versions of grffile with XeLaTeX
    \@ifpackagelater{grffile}{2019/11/01}
    {
      % Do nothing on new versions
    }
    {
      \def\Gread@@xetex#1{%
        \IfFileExists{"\Gin@base".bb}%
        {\Gread@eps{\Gin@base.bb}}%
        {\Gread@@xetex@aux#1}%
      }
    }
    \makeatother
    \usepackage[Export]{adjustbox} % Used to constrain images to a maximum size
    \adjustboxset{max size={0.9\linewidth}{0.9\paperheight}}

    % The hyperref package gives us a pdf with properly built
    % internal navigation ('pdf bookmarks' for the table of contents,
    % internal cross-reference links, web links for URLs, etc.)
    \usepackage{hyperref}
    % The default LaTeX title has an obnoxious amount of whitespace. By default,
    % titling removes some of it. It also provides customization options.
    \usepackage{titling}
    \usepackage{longtable} % longtable support required by pandoc >1.10
    \usepackage{booktabs}  % table support for pandoc > 1.12.2
    \usepackage[inline]{enumitem} % IRkernel/repr support (it uses the enumerate* environment)
    \usepackage[normalem]{ulem} % ulem is needed to support strikethroughs (\sout)
                                % normalem makes italics be italics, not underlines
    \usepackage{mathrsfs}
    

    
    % Colors for the hyperref package
    \definecolor{urlcolor}{rgb}{0,.145,.698}
    \definecolor{linkcolor}{rgb}{.71,0.21,0.01}
    \definecolor{citecolor}{rgb}{.12,.54,.11}

    % ANSI colors
    \definecolor{ansi-black}{HTML}{3E424D}
    \definecolor{ansi-black-intense}{HTML}{282C36}
    \definecolor{ansi-red}{HTML}{E75C58}
    \definecolor{ansi-red-intense}{HTML}{B22B31}
    \definecolor{ansi-green}{HTML}{00A250}
    \definecolor{ansi-green-intense}{HTML}{007427}
    \definecolor{ansi-yellow}{HTML}{DDB62B}
    \definecolor{ansi-yellow-intense}{HTML}{B27D12}
    \definecolor{ansi-blue}{HTML}{208FFB}
    \definecolor{ansi-blue-intense}{HTML}{0065CA}
    \definecolor{ansi-magenta}{HTML}{D160C4}
    \definecolor{ansi-magenta-intense}{HTML}{A03196}
    \definecolor{ansi-cyan}{HTML}{60C6C8}
    \definecolor{ansi-cyan-intense}{HTML}{258F8F}
    \definecolor{ansi-white}{HTML}{C5C1B4}
    \definecolor{ansi-white-intense}{HTML}{A1A6B2}
    \definecolor{ansi-default-inverse-fg}{HTML}{FFFFFF}
    \definecolor{ansi-default-inverse-bg}{HTML}{000000}

    % common color for the border for error outputs.
    \definecolor{outerrorbackground}{HTML}{FFDFDF}

    % commands and environments needed by pandoc snippets
    % extracted from the output of `pandoc -s`
    \providecommand{\tightlist}{%
      \setlength{\itemsep}{0pt}\setlength{\parskip}{0pt}}
    \DefineVerbatimEnvironment{Highlighting}{Verbatim}{commandchars=\\\{\}}
    % Add ',fontsize=\small' for more characters per line
    \newenvironment{Shaded}{}{}
    \newcommand{\KeywordTok}[1]{\textcolor[rgb]{0.00,0.44,0.13}{\textbf{{#1}}}}
    \newcommand{\DataTypeTok}[1]{\textcolor[rgb]{0.56,0.13,0.00}{{#1}}}
    \newcommand{\DecValTok}[1]{\textcolor[rgb]{0.25,0.63,0.44}{{#1}}}
    \newcommand{\BaseNTok}[1]{\textcolor[rgb]{0.25,0.63,0.44}{{#1}}}
    \newcommand{\FloatTok}[1]{\textcolor[rgb]{0.25,0.63,0.44}{{#1}}}
    \newcommand{\CharTok}[1]{\textcolor[rgb]{0.25,0.44,0.63}{{#1}}}
    \newcommand{\StringTok}[1]{\textcolor[rgb]{0.25,0.44,0.63}{{#1}}}
    \newcommand{\CommentTok}[1]{\textcolor[rgb]{0.38,0.63,0.69}{\textit{{#1}}}}
    \newcommand{\OtherTok}[1]{\textcolor[rgb]{0.00,0.44,0.13}{{#1}}}
    \newcommand{\AlertTok}[1]{\textcolor[rgb]{1.00,0.00,0.00}{\textbf{{#1}}}}
    \newcommand{\FunctionTok}[1]{\textcolor[rgb]{0.02,0.16,0.49}{{#1}}}
    \newcommand{\RegionMarkerTok}[1]{{#1}}
    \newcommand{\ErrorTok}[1]{\textcolor[rgb]{1.00,0.00,0.00}{\textbf{{#1}}}}
    \newcommand{\NormalTok}[1]{{#1}}
    
    % Additional commands for more recent versions of Pandoc
    \newcommand{\ConstantTok}[1]{\textcolor[rgb]{0.53,0.00,0.00}{{#1}}}
    \newcommand{\SpecialCharTok}[1]{\textcolor[rgb]{0.25,0.44,0.63}{{#1}}}
    \newcommand{\VerbatimStringTok}[1]{\textcolor[rgb]{0.25,0.44,0.63}{{#1}}}
    \newcommand{\SpecialStringTok}[1]{\textcolor[rgb]{0.73,0.40,0.53}{{#1}}}
    \newcommand{\ImportTok}[1]{{#1}}
    \newcommand{\DocumentationTok}[1]{\textcolor[rgb]{0.73,0.13,0.13}{\textit{{#1}}}}
    \newcommand{\AnnotationTok}[1]{\textcolor[rgb]{0.38,0.63,0.69}{\textbf{\textit{{#1}}}}}
    \newcommand{\CommentVarTok}[1]{\textcolor[rgb]{0.38,0.63,0.69}{\textbf{\textit{{#1}}}}}
    \newcommand{\VariableTok}[1]{\textcolor[rgb]{0.10,0.09,0.49}{{#1}}}
    \newcommand{\ControlFlowTok}[1]{\textcolor[rgb]{0.00,0.44,0.13}{\textbf{{#1}}}}
    \newcommand{\OperatorTok}[1]{\textcolor[rgb]{0.40,0.40,0.40}{{#1}}}
    \newcommand{\BuiltInTok}[1]{{#1}}
    \newcommand{\ExtensionTok}[1]{{#1}}
    \newcommand{\PreprocessorTok}[1]{\textcolor[rgb]{0.74,0.48,0.00}{{#1}}}
    \newcommand{\AttributeTok}[1]{\textcolor[rgb]{0.49,0.56,0.16}{{#1}}}
    \newcommand{\InformationTok}[1]{\textcolor[rgb]{0.38,0.63,0.69}{\textbf{\textit{{#1}}}}}
    \newcommand{\WarningTok}[1]{\textcolor[rgb]{0.38,0.63,0.69}{\textbf{\textit{{#1}}}}}
    
    
    % Define a nice break command that doesn't care if a line doesn't already
    % exist.
    \def\br{\hspace*{\fill} \\* }
    % Math Jax compatibility definitions
    \def\gt{>}
    \def\lt{<}
    \let\Oldtex\TeX
    \let\Oldlatex\LaTeX
    \renewcommand{\TeX}{\textrm{\Oldtex}}
    \renewcommand{\LaTeX}{\textrm{\Oldlatex}}
    % Document parameters
    % Document title
    \title{session\_1}
    
    
    
    
    
% Pygments definitions
\makeatletter
\def\PY@reset{\let\PY@it=\relax \let\PY@bf=\relax%
    \let\PY@ul=\relax \let\PY@tc=\relax%
    \let\PY@bc=\relax \let\PY@ff=\relax}
\def\PY@tok#1{\csname PY@tok@#1\endcsname}
\def\PY@toks#1+{\ifx\relax#1\empty\else%
    \PY@tok{#1}\expandafter\PY@toks\fi}
\def\PY@do#1{\PY@bc{\PY@tc{\PY@ul{%
    \PY@it{\PY@bf{\PY@ff{#1}}}}}}}
\def\PY#1#2{\PY@reset\PY@toks#1+\relax+\PY@do{#2}}

\expandafter\def\csname PY@tok@w\endcsname{\def\PY@tc##1{\textcolor[rgb]{0.73,0.73,0.73}{##1}}}
\expandafter\def\csname PY@tok@c\endcsname{\let\PY@it=\textit\def\PY@tc##1{\textcolor[rgb]{0.25,0.50,0.50}{##1}}}
\expandafter\def\csname PY@tok@cp\endcsname{\def\PY@tc##1{\textcolor[rgb]{0.74,0.48,0.00}{##1}}}
\expandafter\def\csname PY@tok@k\endcsname{\let\PY@bf=\textbf\def\PY@tc##1{\textcolor[rgb]{0.00,0.50,0.00}{##1}}}
\expandafter\def\csname PY@tok@kp\endcsname{\def\PY@tc##1{\textcolor[rgb]{0.00,0.50,0.00}{##1}}}
\expandafter\def\csname PY@tok@kt\endcsname{\def\PY@tc##1{\textcolor[rgb]{0.69,0.00,0.25}{##1}}}
\expandafter\def\csname PY@tok@o\endcsname{\def\PY@tc##1{\textcolor[rgb]{0.40,0.40,0.40}{##1}}}
\expandafter\def\csname PY@tok@ow\endcsname{\let\PY@bf=\textbf\def\PY@tc##1{\textcolor[rgb]{0.67,0.13,1.00}{##1}}}
\expandafter\def\csname PY@tok@nb\endcsname{\def\PY@tc##1{\textcolor[rgb]{0.00,0.50,0.00}{##1}}}
\expandafter\def\csname PY@tok@nf\endcsname{\def\PY@tc##1{\textcolor[rgb]{0.00,0.00,1.00}{##1}}}
\expandafter\def\csname PY@tok@nc\endcsname{\let\PY@bf=\textbf\def\PY@tc##1{\textcolor[rgb]{0.00,0.00,1.00}{##1}}}
\expandafter\def\csname PY@tok@nn\endcsname{\let\PY@bf=\textbf\def\PY@tc##1{\textcolor[rgb]{0.00,0.00,1.00}{##1}}}
\expandafter\def\csname PY@tok@ne\endcsname{\let\PY@bf=\textbf\def\PY@tc##1{\textcolor[rgb]{0.82,0.25,0.23}{##1}}}
\expandafter\def\csname PY@tok@nv\endcsname{\def\PY@tc##1{\textcolor[rgb]{0.10,0.09,0.49}{##1}}}
\expandafter\def\csname PY@tok@no\endcsname{\def\PY@tc##1{\textcolor[rgb]{0.53,0.00,0.00}{##1}}}
\expandafter\def\csname PY@tok@nl\endcsname{\def\PY@tc##1{\textcolor[rgb]{0.63,0.63,0.00}{##1}}}
\expandafter\def\csname PY@tok@ni\endcsname{\let\PY@bf=\textbf\def\PY@tc##1{\textcolor[rgb]{0.60,0.60,0.60}{##1}}}
\expandafter\def\csname PY@tok@na\endcsname{\def\PY@tc##1{\textcolor[rgb]{0.49,0.56,0.16}{##1}}}
\expandafter\def\csname PY@tok@nt\endcsname{\let\PY@bf=\textbf\def\PY@tc##1{\textcolor[rgb]{0.00,0.50,0.00}{##1}}}
\expandafter\def\csname PY@tok@nd\endcsname{\def\PY@tc##1{\textcolor[rgb]{0.67,0.13,1.00}{##1}}}
\expandafter\def\csname PY@tok@s\endcsname{\def\PY@tc##1{\textcolor[rgb]{0.73,0.13,0.13}{##1}}}
\expandafter\def\csname PY@tok@sd\endcsname{\let\PY@it=\textit\def\PY@tc##1{\textcolor[rgb]{0.73,0.13,0.13}{##1}}}
\expandafter\def\csname PY@tok@si\endcsname{\let\PY@bf=\textbf\def\PY@tc##1{\textcolor[rgb]{0.73,0.40,0.53}{##1}}}
\expandafter\def\csname PY@tok@se\endcsname{\let\PY@bf=\textbf\def\PY@tc##1{\textcolor[rgb]{0.73,0.40,0.13}{##1}}}
\expandafter\def\csname PY@tok@sr\endcsname{\def\PY@tc##1{\textcolor[rgb]{0.73,0.40,0.53}{##1}}}
\expandafter\def\csname PY@tok@ss\endcsname{\def\PY@tc##1{\textcolor[rgb]{0.10,0.09,0.49}{##1}}}
\expandafter\def\csname PY@tok@sx\endcsname{\def\PY@tc##1{\textcolor[rgb]{0.00,0.50,0.00}{##1}}}
\expandafter\def\csname PY@tok@m\endcsname{\def\PY@tc##1{\textcolor[rgb]{0.40,0.40,0.40}{##1}}}
\expandafter\def\csname PY@tok@gh\endcsname{\let\PY@bf=\textbf\def\PY@tc##1{\textcolor[rgb]{0.00,0.00,0.50}{##1}}}
\expandafter\def\csname PY@tok@gu\endcsname{\let\PY@bf=\textbf\def\PY@tc##1{\textcolor[rgb]{0.50,0.00,0.50}{##1}}}
\expandafter\def\csname PY@tok@gd\endcsname{\def\PY@tc##1{\textcolor[rgb]{0.63,0.00,0.00}{##1}}}
\expandafter\def\csname PY@tok@gi\endcsname{\def\PY@tc##1{\textcolor[rgb]{0.00,0.63,0.00}{##1}}}
\expandafter\def\csname PY@tok@gr\endcsname{\def\PY@tc##1{\textcolor[rgb]{1.00,0.00,0.00}{##1}}}
\expandafter\def\csname PY@tok@ge\endcsname{\let\PY@it=\textit}
\expandafter\def\csname PY@tok@gs\endcsname{\let\PY@bf=\textbf}
\expandafter\def\csname PY@tok@gp\endcsname{\let\PY@bf=\textbf\def\PY@tc##1{\textcolor[rgb]{0.00,0.00,0.50}{##1}}}
\expandafter\def\csname PY@tok@go\endcsname{\def\PY@tc##1{\textcolor[rgb]{0.53,0.53,0.53}{##1}}}
\expandafter\def\csname PY@tok@gt\endcsname{\def\PY@tc##1{\textcolor[rgb]{0.00,0.27,0.87}{##1}}}
\expandafter\def\csname PY@tok@err\endcsname{\def\PY@bc##1{\setlength{\fboxsep}{0pt}\fcolorbox[rgb]{1.00,0.00,0.00}{1,1,1}{\strut ##1}}}
\expandafter\def\csname PY@tok@kc\endcsname{\let\PY@bf=\textbf\def\PY@tc##1{\textcolor[rgb]{0.00,0.50,0.00}{##1}}}
\expandafter\def\csname PY@tok@kd\endcsname{\let\PY@bf=\textbf\def\PY@tc##1{\textcolor[rgb]{0.00,0.50,0.00}{##1}}}
\expandafter\def\csname PY@tok@kn\endcsname{\let\PY@bf=\textbf\def\PY@tc##1{\textcolor[rgb]{0.00,0.50,0.00}{##1}}}
\expandafter\def\csname PY@tok@kr\endcsname{\let\PY@bf=\textbf\def\PY@tc##1{\textcolor[rgb]{0.00,0.50,0.00}{##1}}}
\expandafter\def\csname PY@tok@bp\endcsname{\def\PY@tc##1{\textcolor[rgb]{0.00,0.50,0.00}{##1}}}
\expandafter\def\csname PY@tok@fm\endcsname{\def\PY@tc##1{\textcolor[rgb]{0.00,0.00,1.00}{##1}}}
\expandafter\def\csname PY@tok@vc\endcsname{\def\PY@tc##1{\textcolor[rgb]{0.10,0.09,0.49}{##1}}}
\expandafter\def\csname PY@tok@vg\endcsname{\def\PY@tc##1{\textcolor[rgb]{0.10,0.09,0.49}{##1}}}
\expandafter\def\csname PY@tok@vi\endcsname{\def\PY@tc##1{\textcolor[rgb]{0.10,0.09,0.49}{##1}}}
\expandafter\def\csname PY@tok@vm\endcsname{\def\PY@tc##1{\textcolor[rgb]{0.10,0.09,0.49}{##1}}}
\expandafter\def\csname PY@tok@sa\endcsname{\def\PY@tc##1{\textcolor[rgb]{0.73,0.13,0.13}{##1}}}
\expandafter\def\csname PY@tok@sb\endcsname{\def\PY@tc##1{\textcolor[rgb]{0.73,0.13,0.13}{##1}}}
\expandafter\def\csname PY@tok@sc\endcsname{\def\PY@tc##1{\textcolor[rgb]{0.73,0.13,0.13}{##1}}}
\expandafter\def\csname PY@tok@dl\endcsname{\def\PY@tc##1{\textcolor[rgb]{0.73,0.13,0.13}{##1}}}
\expandafter\def\csname PY@tok@s2\endcsname{\def\PY@tc##1{\textcolor[rgb]{0.73,0.13,0.13}{##1}}}
\expandafter\def\csname PY@tok@sh\endcsname{\def\PY@tc##1{\textcolor[rgb]{0.73,0.13,0.13}{##1}}}
\expandafter\def\csname PY@tok@s1\endcsname{\def\PY@tc##1{\textcolor[rgb]{0.73,0.13,0.13}{##1}}}
\expandafter\def\csname PY@tok@mb\endcsname{\def\PY@tc##1{\textcolor[rgb]{0.40,0.40,0.40}{##1}}}
\expandafter\def\csname PY@tok@mf\endcsname{\def\PY@tc##1{\textcolor[rgb]{0.40,0.40,0.40}{##1}}}
\expandafter\def\csname PY@tok@mh\endcsname{\def\PY@tc##1{\textcolor[rgb]{0.40,0.40,0.40}{##1}}}
\expandafter\def\csname PY@tok@mi\endcsname{\def\PY@tc##1{\textcolor[rgb]{0.40,0.40,0.40}{##1}}}
\expandafter\def\csname PY@tok@il\endcsname{\def\PY@tc##1{\textcolor[rgb]{0.40,0.40,0.40}{##1}}}
\expandafter\def\csname PY@tok@mo\endcsname{\def\PY@tc##1{\textcolor[rgb]{0.40,0.40,0.40}{##1}}}
\expandafter\def\csname PY@tok@ch\endcsname{\let\PY@it=\textit\def\PY@tc##1{\textcolor[rgb]{0.25,0.50,0.50}{##1}}}
\expandafter\def\csname PY@tok@cm\endcsname{\let\PY@it=\textit\def\PY@tc##1{\textcolor[rgb]{0.25,0.50,0.50}{##1}}}
\expandafter\def\csname PY@tok@cpf\endcsname{\let\PY@it=\textit\def\PY@tc##1{\textcolor[rgb]{0.25,0.50,0.50}{##1}}}
\expandafter\def\csname PY@tok@c1\endcsname{\let\PY@it=\textit\def\PY@tc##1{\textcolor[rgb]{0.25,0.50,0.50}{##1}}}
\expandafter\def\csname PY@tok@cs\endcsname{\let\PY@it=\textit\def\PY@tc##1{\textcolor[rgb]{0.25,0.50,0.50}{##1}}}

\def\PYZbs{\char`\\}
\def\PYZus{\char`\_}
\def\PYZob{\char`\{}
\def\PYZcb{\char`\}}
\def\PYZca{\char`\^}
\def\PYZam{\char`\&}
\def\PYZlt{\char`\<}
\def\PYZgt{\char`\>}
\def\PYZsh{\char`\#}
\def\PYZpc{\char`\%}
\def\PYZdl{\char`\$}
\def\PYZhy{\char`\-}
\def\PYZsq{\char`\'}
\def\PYZdq{\char`\"}
\def\PYZti{\char`\~}
% for compatibility with earlier versions
\def\PYZat{@}
\def\PYZlb{[}
\def\PYZrb{]}
\makeatother


    % For linebreaks inside Verbatim environment from package fancyvrb. 
    \makeatletter
        \newbox\Wrappedcontinuationbox 
        \newbox\Wrappedvisiblespacebox 
        \newcommand*\Wrappedvisiblespace {\textcolor{red}{\textvisiblespace}} 
        \newcommand*\Wrappedcontinuationsymbol {\textcolor{red}{\llap{\tiny$\m@th\hookrightarrow$}}} 
        \newcommand*\Wrappedcontinuationindent {3ex } 
        \newcommand*\Wrappedafterbreak {\kern\Wrappedcontinuationindent\copy\Wrappedcontinuationbox} 
        % Take advantage of the already applied Pygments mark-up to insert 
        % potential linebreaks for TeX processing. 
        %        {, <, #, %, $, ' and ": go to next line. 
        %        _, }, ^, &, >, - and ~: stay at end of broken line. 
        % Use of \textquotesingle for straight quote. 
        \newcommand*\Wrappedbreaksatspecials {% 
            \def\PYGZus{\discretionary{\char`\_}{\Wrappedafterbreak}{\char`\_}}% 
            \def\PYGZob{\discretionary{}{\Wrappedafterbreak\char`\{}{\char`\{}}% 
            \def\PYGZcb{\discretionary{\char`\}}{\Wrappedafterbreak}{\char`\}}}% 
            \def\PYGZca{\discretionary{\char`\^}{\Wrappedafterbreak}{\char`\^}}% 
            \def\PYGZam{\discretionary{\char`\&}{\Wrappedafterbreak}{\char`\&}}% 
            \def\PYGZlt{\discretionary{}{\Wrappedafterbreak\char`\<}{\char`\<}}% 
            \def\PYGZgt{\discretionary{\char`\>}{\Wrappedafterbreak}{\char`\>}}% 
            \def\PYGZsh{\discretionary{}{\Wrappedafterbreak\char`\#}{\char`\#}}% 
            \def\PYGZpc{\discretionary{}{\Wrappedafterbreak\char`\%}{\char`\%}}% 
            \def\PYGZdl{\discretionary{}{\Wrappedafterbreak\char`\$}{\char`\$}}% 
            \def\PYGZhy{\discretionary{\char`\-}{\Wrappedafterbreak}{\char`\-}}% 
            \def\PYGZsq{\discretionary{}{\Wrappedafterbreak\textquotesingle}{\textquotesingle}}% 
            \def\PYGZdq{\discretionary{}{\Wrappedafterbreak\char`\"}{\char`\"}}% 
            \def\PYGZti{\discretionary{\char`\~}{\Wrappedafterbreak}{\char`\~}}% 
        } 
        % Some characters . , ; ? ! / are not pygmentized. 
        % This macro makes them "active" and they will insert potential linebreaks 
        \newcommand*\Wrappedbreaksatpunct {% 
            \lccode`\~`\.\lowercase{\def~}{\discretionary{\hbox{\char`\.}}{\Wrappedafterbreak}{\hbox{\char`\.}}}% 
            \lccode`\~`\,\lowercase{\def~}{\discretionary{\hbox{\char`\,}}{\Wrappedafterbreak}{\hbox{\char`\,}}}% 
            \lccode`\~`\;\lowercase{\def~}{\discretionary{\hbox{\char`\;}}{\Wrappedafterbreak}{\hbox{\char`\;}}}% 
            \lccode`\~`\:\lowercase{\def~}{\discretionary{\hbox{\char`\:}}{\Wrappedafterbreak}{\hbox{\char`\:}}}% 
            \lccode`\~`\?\lowercase{\def~}{\discretionary{\hbox{\char`\?}}{\Wrappedafterbreak}{\hbox{\char`\?}}}% 
            \lccode`\~`\!\lowercase{\def~}{\discretionary{\hbox{\char`\!}}{\Wrappedafterbreak}{\hbox{\char`\!}}}% 
            \lccode`\~`\/\lowercase{\def~}{\discretionary{\hbox{\char`\/}}{\Wrappedafterbreak}{\hbox{\char`\/}}}% 
            \catcode`\.\active
            \catcode`\,\active 
            \catcode`\;\active
            \catcode`\:\active
            \catcode`\?\active
            \catcode`\!\active
            \catcode`\/\active 
            \lccode`\~`\~ 	
        }
    \makeatother

    \let\OriginalVerbatim=\Verbatim
    \makeatletter
    \renewcommand{\Verbatim}[1][1]{%
        %\parskip\z@skip
        \sbox\Wrappedcontinuationbox {\Wrappedcontinuationsymbol}%
        \sbox\Wrappedvisiblespacebox {\FV@SetupFont\Wrappedvisiblespace}%
        \def\FancyVerbFormatLine ##1{\hsize\linewidth
            \vtop{\raggedright\hyphenpenalty\z@\exhyphenpenalty\z@
                \doublehyphendemerits\z@\finalhyphendemerits\z@
                \strut ##1\strut}%
        }%
        % If the linebreak is at a space, the latter will be displayed as visible
        % space at end of first line, and a continuation symbol starts next line.
        % Stretch/shrink are however usually zero for typewriter font.
        \def\FV@Space {%
            \nobreak\hskip\z@ plus\fontdimen3\font minus\fontdimen4\font
            \discretionary{\copy\Wrappedvisiblespacebox}{\Wrappedafterbreak}
            {\kern\fontdimen2\font}%
        }%
        
        % Allow breaks at special characters using \PYG... macros.
        \Wrappedbreaksatspecials
        % Breaks at punctuation characters . , ; ? ! and / need catcode=\active 	
        \OriginalVerbatim[#1,codes*=\Wrappedbreaksatpunct]%
    }
    \makeatother

    % Exact colors from NB
    \definecolor{incolor}{HTML}{303F9F}
    \definecolor{outcolor}{HTML}{D84315}
    \definecolor{cellborder}{HTML}{CFCFCF}
    \definecolor{cellbackground}{HTML}{F7F7F7}
    
    % prompt
    \makeatletter
    \newcommand{\boxspacing}{\kern\kvtcb@left@rule\kern\kvtcb@boxsep}
    \makeatother
    \newcommand{\prompt}[4]{
        {\ttfamily\llap{{\color{#2}[#3]:\hspace{3pt}#4}}\vspace{-\baselineskip}}
    }
    

    
    % Prevent overflowing lines due to hard-to-break entities
    \sloppy 
    % Setup hyperref package
    \hypersetup{
      breaklinks=true,  % so long urls are correctly broken across lines
      colorlinks=true,
      urlcolor=urlcolor,
      linkcolor=linkcolor,
      citecolor=citecolor,
      }
    % Slightly bigger margins than the latex defaults
    
    \geometry{verbose,tmargin=1in,bmargin=1in,lmargin=1in,rmargin=1in}
    
    

\begin{document}
    
    \maketitle
    
    

    
    \hypertarget{working-with-sql-databases}{%
\section{Working with SQL databases}\label{working-with-sql-databases}}

\textbf{Author:} `Felipe Millacura' \textbf{Date:} `6th December 2020'

\hypertarget{learning-objectives}{%
\subsection{Learning Objectives}\label{learning-objectives}}

\begin{itemize}
\tightlist
\item
  Understanding how SQL databases work
\item
  Be able to retrieve information from an SQL database
\end{itemize}

\hypertarget{introduction}{%
\subsection{Introduction}\label{introduction}}

\hypertarget{what-is-a-database}{%
\subsubsection{What is a database?}\label{what-is-a-database}}

A database is just somewhere for us to store our data. There are many
different shapes and sizes of database. SQL is a language which is often
used to query these databases and that's what we will be learning today.

\hypertarget{what-do-we-do-with-databases}{%
\subsubsection{What do we do with
databases?}\label{what-do-we-do-with-databases}}

Task - (2 mins) Have a think about it and give some examples of data you
might store in a database.

Solution

The possibilities are truly vast: customers and customer orders. Climate
sampling sites and climate data gathered there.

What sorts of manipulations do we make to data in databases?

\begin{itemize}
\tightlist
\item
  \textbf{Create} (at some point, usually before we are involved, data
  goes into a database)
\item
  \textbf{Read} (we need to get data out to perform analyses)
\item
  \textbf{Update} (occasionally, we may need to be able to change data)
\item
  \textbf{Delete} (occasionally, we need to be able to remove data from
  our database)
\end{itemize}

We refer to these four operations as \textbf{``CRUD''}. As data
analysts, we are naturally most interested in \textbf{Read}ing from
databases: we want to import data into an analysis environment and
produce useful and actionable insights. But, we'll briefly show you how
to perform the other three operations, just so you have seen them in
action!

    \hypertarget{why-databases-are-needed}{%
\subsubsection{Why databases are
needed?}\label{why-databases-are-needed}}

\begin{enumerate}
\def\labelenumi{\arabic{enumi}.}
\tightlist
\item
  Databases \emph{can store very large numbers of records efficiently}
  (they take up little space).
\item
  It is \emph{very quick} and easy to find information.
\item
  It is \emph{easy to add new data and to edit or delete} old data.
\item
  Data \emph{can be searched easily}, eg `find all Ford cars'.
\item
  Data \emph{can be sorted easily}, for example into `date first
  registered' order.
\item
  Data \emph{can be imported into other applications}, for example a
  mail-merge letter to a customer saying that an MOT test is due.
\item
  \emph{More than one person can access the same database} at the same
  time - multi-access.
\item
  \emph{Security may be better} than in paper files.
\end{enumerate}

My company uses Excel and it's just fine

\begin{longtable}[]{@{}ll@{}}
\toprule
Feature & Maximum limit\tabularnewline
\midrule
\endhead
Open workbooks & Limited by available memory and system
resources\tabularnewline
Total number of rows on a worksheet & 1,048,576 rows\tabularnewline
Total number of columns on a worksheet & 16,384 columns\tabularnewline
\bottomrule
\end{longtable}

SOURCE: https://www.bbc.com/news/technology-54423988

    \hypertarget{what-is-sql}{%
\subsubsection{What is SQL?}\label{what-is-sql}}

``SQL'' stands for ``Structured Query Language'' (pronounced either as
``ess-queue-ell'' or ``sequel''). There are number of different
`versions' of SQL in common use: some freely available and others
proprietary. You should be aware that there are lots of different
versions of SQL out there: PostgreSQL, MySQL, Oracle, SQL Server
etc\ldots{}

The core functionality of each of the versions tends to be the same, and
it's really only core functionality we'll be looking at here!

\hypertarget{database-structure}{%
\subsubsection{Database structure}\label{database-structure}}

In SQL, a database is a collection of \textbf{tables}. A table is a
collection of \textbf{columns} and \textbf{rows}.

\begin{itemize}
\tightlist
\item
  A table describes the type of item that we want to store.
\item
  A column represents some information we might find interesting about
  that item.
\item
  A row is the physical data we want to save.
\end{itemize}

For example, we might have a \texttt{zoo} database with a table called
\texttt{animals}. The \texttt{animals} table might have the columns
\texttt{name}, \texttt{age} and \texttt{species}, and we'll also add in
an \texttt{animal\_id} column to keep track of individual animals. The
\texttt{animals} table would then have rows of data like:

\begin{longtable}[]{@{}llll@{}}
\toprule
\texttt{animal\_id} & \texttt{name} & \texttt{age} &
\texttt{species}\tabularnewline
\midrule
\endhead
1 & Leo & 12 & Lion\tabularnewline
2 & Tony & 8 & Tiger\tabularnewline
3 & Matilda & 6 & Cow\tabularnewline
4 & Winnie & 12 & Bear\tabularnewline
\bottomrule
\end{longtable}

    \hypertarget{table-relationships}{%
\subsubsection{Table relationships}\label{table-relationships}}

Often the tables in a database are \textbf{related} to each other in
some fashion, and in fact people often use the terms `SQL database' and
`relational database' interchangeably, as virtually all relational
databases use SQL.

Let's see an example of a relationship. Imagine we expand our
\texttt{zoo} database by adding a \texttt{diets} table to help the
zookeepers track what to feed each animal.

\begin{longtable}[]{@{}ll@{}}
\toprule
\texttt{diet\_id} & \texttt{diet\_type}\tabularnewline
\midrule
\endhead
1 & herbivore\tabularnewline
2 & carnivore\tabularnewline
3 & omnivore\tabularnewline
\bottomrule
\end{longtable}

So far so good. But now we need to say \textbf{which diet each animal
should receive}.

Think of the two tables we have in the database so far: \texttt{animals}
and \texttt{diets}. Can you see how to change the database to indicate
which diet each animal in the \texttt{zoo} should receive?

\textbf{Hint:} We need to add something to the \texttt{animals} table.
Can you think what?

Solution

We need to \textbf{add an extra column} to the \texttt{animals} table,
like so

\begin{longtable}[]{@{}lllll@{}}
\toprule
\texttt{animal\_id} & \texttt{name} & \texttt{age} & \texttt{species} &
\texttt{diet\_id}\tabularnewline
\midrule
\endhead
1 & Leo & 12 & Lion & 2\tabularnewline
2 & Tony & 8 & Tiger & 2\tabularnewline
3 & Matilda & 6 & Cow & 1\tabularnewline
4 & Bernice & 12 & Bear & 3\tabularnewline
\bottomrule
\end{longtable}

This establishes a \textbf{relationship} between the two tables! Every
row in the \texttt{animals} table is now linked to a row in the
\texttt{diets} table.

    \hypertarget{creating-a-database}{%
\subsubsection{Creating a database}\label{creating-a-database}}

Before we can do anything though, we need to create a table to store our
records in. But before we can create a table, we have to create a
database to put it in!

\begin{Shaded}
\begin{Highlighting}[]
\CommentTok{{-}{-} sql terminal}
\KeywordTok{CREATE} \KeywordTok{DATABASE}\NormalTok{ database\_name; }\CommentTok{{-}{-} REMEMBER SEMI COLON}
\end{Highlighting}
\end{Shaded}

For example

\begin{Shaded}
\begin{Highlighting}[]
\CommentTok{{-}{-} udacity\_students.sql}

\KeywordTok{CREATE} \KeywordTok{DATABASE}\NormalTok{ udacity\_students;}
\end{Highlighting}
\end{Shaded}

    \hypertarget{creating-tables}{%
\subsubsection{Creating tables}\label{creating-tables}}

By convention, we will name our database tables as the plural of the
thing we are creating. So rows of animal data would be stored in a table
called \texttt{animals}. Sheep would be stored in a table called\ldots{}
well, \texttt{sheep}, but you might want to call it \texttt{sheeps} to
make it clear it's a table holding data on multiple sheep. Sometimes
grammar has to suffer for technical clarity!

We can use the following template every time we create a new database
table:

\begin{Shaded}
\begin{Highlighting}[]

\KeywordTok{CREATE} \KeywordTok{TABLE}\NormalTok{ table\_name (}
\NormalTok{  column\_name1 DATA\_TYPE,}
\NormalTok{  column\_name2 DATA\_TYPE,}
\NormalTok{  column\_name3 DATA\_TYPE}
\NormalTok{)}
\end{Highlighting}
\end{Shaded}

    So before we run off and create lots of shiny tables, we need to talk
about datatypes. You'll be glad to hear they roughly match up to what we
have already seen in R. There are many data types we can use in SQL -
the most common we will be using are:

\begin{itemize}
\tightlist
\item
  VARCHAR - fixed length text (string)
\item
  INT - integer numerical data (4-byte integer)
\item
  REAL - continuously valued numerical data (8-byte floating point)
\item
  BOOLEAN - true / false data (TRUE, FALSE, booleans)
\item
  TIMESTAMP - date and time information
\end{itemize}

We can pass arguments to VARCHAR to say how large we want the data in
the field to be as a maximum.

Let's continue with the previous example

\begin{Shaded}
\begin{Highlighting}[]
\CommentTok{{-}{-} udacity\_students.sql}
\KeywordTok{CREATE} \KeywordTok{TABLE}\NormalTok{ students (}
\NormalTok{  name }\DataTypeTok{VARCHAR}\NormalTok{(}\DecValTok{255}\NormalTok{),}
\NormalTok{  on\_track }\DataTypeTok{BOOLEAN}\NormalTok{,}
\NormalTok{  points  }\DataTypeTok{INT}\NormalTok{,}
\NormalTok{  connected }\DataTypeTok{TIMESTAMP}

\NormalTok{)}
\end{Highlighting}
\end{Shaded}

    \hypertarget{reading-c-r-ud}{%
\subsubsection{\texorpdfstring{Reading
(C-\textbf{R}-UD)}{Reading (C-R-UD)}}\label{reading-c-r-ud}}

This is the R in CRUD. We give only a brief introduction here, as a more
detailed lesson is coming up!

We have been `reading' records with the \texttt{SELECT} command.

\begin{Shaded}
\begin{Highlighting}[]
\CommentTok{{-}{-} udacity\_students.sql}
\KeywordTok{SELECT} \OperatorTok{*} 
\KeywordTok{FROM}\NormalTok{ students;}
\end{Highlighting}
\end{Shaded}

The star tells SQL that we want \textbf{all} of the fields returned. If
we instead said:

\begin{Shaded}
\begin{Highlighting}[]
\CommentTok{{-}{-} udacity\_students.sql}
\KeywordTok{SELECT}\NormalTok{ name }
\KeywordTok{FROM}\NormalTok{ students;}
\end{Highlighting}
\end{Shaded}

then only the \texttt{name}s will be returned by the query.

You can also subset a table based on a condition that \textbf{must} be
met

\begin{Shaded}
\begin{Highlighting}[]
\CommentTok{{-}{-} udacity\_students.sql}
\KeywordTok{SELECT}\NormalTok{ name }
\KeywordTok{FROM}\NormalTok{ students}
\KeywordTok{WHERE}\NormalTok{ points }\OperatorTok{\textgreater{}} \DecValTok{500}\NormalTok{;}
\end{Highlighting}
\end{Shaded}

or order the results in \texttt{ASC} or \texttt{DESC} way

\begin{Shaded}
\begin{Highlighting}[]
\CommentTok{{-}{-} udacity\_students.sql}
\KeywordTok{SELECT}\NormalTok{ name }
\KeywordTok{FROM}\NormalTok{ students}
\KeywordTok{WHERE}\NormalTok{ points }\OperatorTok{\textgreater{}} \DecValTok{10}
\KeywordTok{ORDER} \KeywordTok{BY}\NormalTok{ points; }\CommentTok{{-}{-} ASC by default}
\end{Highlighting}
\end{Shaded}

what if we want just the top 5?

\begin{Shaded}
\begin{Highlighting}[]
\CommentTok{{-}{-} udacity\_students.sql}
\KeywordTok{SELECT}\NormalTok{ name }
\KeywordTok{FROM}\NormalTok{ students}
\KeywordTok{WHERE}\NormalTok{ points }\OperatorTok{\textgreater{}} \DecValTok{10}
\KeywordTok{ORDER} \KeywordTok{BY}\NormalTok{ points }\KeywordTok{DESC}
\KeywordTok{LIMIT} \DecValTok{5}\NormalTok{;}
\end{Highlighting}
\end{Shaded}

    There are multiple operators you can use

\begin{Shaded}
\begin{Highlighting}[]
\OperatorTok{\textgreater{}} \CommentTok{{-}{-}greater than}
\OperatorTok{\textless{}} \CommentTok{{-}{-}less than}
\OperatorTok{\textgreater{}=} \CommentTok{{-}{-}greater or equal to}
\OperatorTok{\textless{}=} \CommentTok{{-}{-}less or equal to}
\OperatorTok{=} \CommentTok{{-}{-}equal to}
\OperatorTok{!=} \CommentTok{{-}{-}not equal to}
\OperatorTok{\textless{}\textgreater{}} \CommentTok{{-}{-}also not equal to (American National Standard Institute {-} ANSI)}
 
\KeywordTok{LIKE} \CommentTok{{-}{-} used in a WHERE clause to search for a specified pattern in a column}
\KeywordTok{AND} \CommentTok{{-}{-} displays a record if all the conditions separated by AND are TRUE.}
\KeywordTok{OR} \CommentTok{{-}{-} displays a record if any of the conditions separated by OR is TRUE}
\KeywordTok{IN} \CommentTok{{-}{-}  allows you to specify multiple values in a WHERE clause}
\KeywordTok{NOT} \CommentTok{{-}{-} displays a record if the condition(s) is NOT TRUE}
\end{Highlighting}
\end{Shaded}

    \hypertarget{updating-cr-u-d}{%
\subsubsection{\texorpdfstring{Updating
(CR-\textbf{U}-D)}{Updating (CR-U-D)}}\label{updating-cr-u-d}}

This is the U in CRUD.

We use the \texttt{UPDATE} command to change the values in existing
records.

Template:

\begin{Shaded}
\begin{Highlighting}[]
\KeywordTok{UPDATE}\NormalTok{ table\_name }\KeywordTok{SET}\NormalTok{ column\_name1 }\OperatorTok{=}\NormalTok{ new\_value1;}
\end{Highlighting}
\end{Shaded}

Let's update the \texttt{on\_track} column to \texttt{TRUE}!

\begin{Shaded}
\begin{Highlighting}[]
\CommentTok{{-}{-} udacity\_students.sql}
\KeywordTok{UPDATE}\NormalTok{ students }\KeywordTok{SET}\NormalTok{ on\_track }\OperatorTok{=} \KeywordTok{TRUE}\NormalTok{;}
\end{Highlighting}
\end{Shaded}

    \hypertarget{deleting-cru-d}{%
\section{\texorpdfstring{Deleting
(CRU-\textbf{D})}{Deleting (CRU-D)}}\label{deleting-cru-d}}

This is the D in CRUD.

To delete records we use the \texttt{DELETE} clause. But \textbf{be
careful}, there's no undo! When a record is deleted from a DB it's gone
for ever. ``Undelete'' in the database world is ``restore from last
night's backup'' (if there \emph{was} a backup\ldots)

Template:

\begin{Shaded}
\begin{Highlighting}[]
\KeywordTok{DELETE} \KeywordTok{FROM}\NormalTok{ table\_name }\KeywordTok{WHERE}\NormalTok{ column\_name }\OperatorTok{=}\NormalTok{ target\_value;}
\end{Highlighting}
\end{Shaded}

\textbf{SPOILERS} Let's delete some students (mean I know!)

\begin{Shaded}
\begin{Highlighting}[]
\CommentTok{{-}{-} udacity\_students.sql}
\KeywordTok{DELETE} \KeywordTok{FROM}\NormalTok{ students }\KeywordTok{WHERE}\NormalTok{ name }\OperatorTok{=} \StringTok{\textquotesingle{}Felipe Millacura\textquotesingle{}}\NormalTok{;}
\KeywordTok{SELECT} \OperatorTok{*} \KeywordTok{FROM}\NormalTok{ students;}
\end{Highlighting}
\end{Shaded}

\textbf{WARNING}: If you don't specify the row(s) with a WHERE clause,
it will delete \emph{everything} in that table!

\begin{Shaded}
\begin{Highlighting}[]
\KeywordTok{DELETE} \KeywordTok{FROM}\NormalTok{ students; }\CommentTok{{-}{-} DELETES EVERYTHING FROM STUDENTS}
\end{Highlighting}
\end{Shaded}

    


    % Add a bibliography block to the postdoc
    
    
    
\end{document}
